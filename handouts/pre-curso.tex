% Options for packages loaded elsewhere
\PassOptionsToPackage{unicode}{hyperref}
\PassOptionsToPackage{hyphens}{url}
\PassOptionsToPackage{dvipsnames,svgnames,x11names}{xcolor}
%
\documentclass[
  a4paper,
]{letter}

\usepackage{amsmath,amssymb}
\usepackage{iftex}
\ifPDFTeX
  \usepackage[T1]{fontenc}
  \usepackage[utf8]{inputenc}
  \usepackage{textcomp} % provide euro and other symbols
\else % if luatex or xetex
  \usepackage{unicode-math}
  \defaultfontfeatures{Scale=MatchLowercase}
  \defaultfontfeatures[\rmfamily]{Ligatures=TeX,Scale=1}
\fi
\usepackage{lmodern}
\ifPDFTeX\else  
    % xetex/luatex font selection
\fi
% Use upquote if available, for straight quotes in verbatim environments
\IfFileExists{upquote.sty}{\usepackage{upquote}}{}
\IfFileExists{microtype.sty}{% use microtype if available
  \usepackage[]{microtype}
  \UseMicrotypeSet[protrusion]{basicmath} % disable protrusion for tt fonts
}{}
\makeatletter
\@ifundefined{KOMAClassName}{% if non-KOMA class
  \IfFileExists{parskip.sty}{%
    \usepackage{parskip}
  }{% else
    \setlength{\parindent}{0pt}
    \setlength{\parskip}{6pt plus 2pt minus 1pt}}
}{% if KOMA class
  \KOMAoptions{parskip=half}}
\makeatother
\usepackage{xcolor}
\setlength{\emergencystretch}{3em} % prevent overfull lines
\setcounter{secnumdepth}{-\maxdimen} % remove section numbering
% Make \paragraph and \subparagraph free-standing
\ifx\paragraph\undefined\else
  \let\oldparagraph\paragraph
  \renewcommand{\paragraph}[1]{\oldparagraph{#1}\mbox{}}
\fi
\ifx\subparagraph\undefined\else
  \let\oldsubparagraph\subparagraph
  \renewcommand{\subparagraph}[1]{\oldsubparagraph{#1}\mbox{}}
\fi

\usepackage{color}
\usepackage{fancyvrb}
\newcommand{\VerbBar}{|}
\newcommand{\VERB}{\Verb[commandchars=\\\{\}]}
\DefineVerbatimEnvironment{Highlighting}{Verbatim}{commandchars=\\\{\}}
% Add ',fontsize=\small' for more characters per line
\usepackage{framed}
\definecolor{shadecolor}{RGB}{241,243,245}
\newenvironment{Shaded}{\begin{snugshade}}{\end{snugshade}}
\newcommand{\AlertTok}[1]{\textcolor[rgb]{0.68,0.00,0.00}{#1}}
\newcommand{\AnnotationTok}[1]{\textcolor[rgb]{0.37,0.37,0.37}{#1}}
\newcommand{\AttributeTok}[1]{\textcolor[rgb]{0.40,0.45,0.13}{#1}}
\newcommand{\BaseNTok}[1]{\textcolor[rgb]{0.68,0.00,0.00}{#1}}
\newcommand{\BuiltInTok}[1]{\textcolor[rgb]{0.00,0.23,0.31}{#1}}
\newcommand{\CharTok}[1]{\textcolor[rgb]{0.13,0.47,0.30}{#1}}
\newcommand{\CommentTok}[1]{\textcolor[rgb]{0.37,0.37,0.37}{#1}}
\newcommand{\CommentVarTok}[1]{\textcolor[rgb]{0.37,0.37,0.37}{\textit{#1}}}
\newcommand{\ConstantTok}[1]{\textcolor[rgb]{0.56,0.35,0.01}{#1}}
\newcommand{\ControlFlowTok}[1]{\textcolor[rgb]{0.00,0.23,0.31}{#1}}
\newcommand{\DataTypeTok}[1]{\textcolor[rgb]{0.68,0.00,0.00}{#1}}
\newcommand{\DecValTok}[1]{\textcolor[rgb]{0.68,0.00,0.00}{#1}}
\newcommand{\DocumentationTok}[1]{\textcolor[rgb]{0.37,0.37,0.37}{\textit{#1}}}
\newcommand{\ErrorTok}[1]{\textcolor[rgb]{0.68,0.00,0.00}{#1}}
\newcommand{\ExtensionTok}[1]{\textcolor[rgb]{0.00,0.23,0.31}{#1}}
\newcommand{\FloatTok}[1]{\textcolor[rgb]{0.68,0.00,0.00}{#1}}
\newcommand{\FunctionTok}[1]{\textcolor[rgb]{0.28,0.35,0.67}{#1}}
\newcommand{\ImportTok}[1]{\textcolor[rgb]{0.00,0.46,0.62}{#1}}
\newcommand{\InformationTok}[1]{\textcolor[rgb]{0.37,0.37,0.37}{#1}}
\newcommand{\KeywordTok}[1]{\textcolor[rgb]{0.00,0.23,0.31}{#1}}
\newcommand{\NormalTok}[1]{\textcolor[rgb]{0.00,0.23,0.31}{#1}}
\newcommand{\OperatorTok}[1]{\textcolor[rgb]{0.37,0.37,0.37}{#1}}
\newcommand{\OtherTok}[1]{\textcolor[rgb]{0.00,0.23,0.31}{#1}}
\newcommand{\PreprocessorTok}[1]{\textcolor[rgb]{0.68,0.00,0.00}{#1}}
\newcommand{\RegionMarkerTok}[1]{\textcolor[rgb]{0.00,0.23,0.31}{#1}}
\newcommand{\SpecialCharTok}[1]{\textcolor[rgb]{0.37,0.37,0.37}{#1}}
\newcommand{\SpecialStringTok}[1]{\textcolor[rgb]{0.13,0.47,0.30}{#1}}
\newcommand{\StringTok}[1]{\textcolor[rgb]{0.13,0.47,0.30}{#1}}
\newcommand{\VariableTok}[1]{\textcolor[rgb]{0.07,0.07,0.07}{#1}}
\newcommand{\VerbatimStringTok}[1]{\textcolor[rgb]{0.13,0.47,0.30}{#1}}
\newcommand{\WarningTok}[1]{\textcolor[rgb]{0.37,0.37,0.37}{\textit{#1}}}

\providecommand{\tightlist}{%
  \setlength{\itemsep}{0pt}\setlength{\parskip}{0pt}}\usepackage{longtable,booktabs,array}
\usepackage{calc} % for calculating minipage widths
% Correct order of tables after \paragraph or \subparagraph
\usepackage{etoolbox}
\makeatletter
\patchcmd\longtable{\par}{\if@noskipsec\mbox{}\fi\par}{}{}
\makeatother
% Allow footnotes in longtable head/foot
\IfFileExists{footnotehyper.sty}{\usepackage{footnotehyper}}{\usepackage{footnote}}
\makesavenoteenv{longtable}
\usepackage{graphicx}
\makeatletter
\def\maxwidth{\ifdim\Gin@nat@width>\linewidth\linewidth\else\Gin@nat@width\fi}
\def\maxheight{\ifdim\Gin@nat@height>\textheight\textheight\else\Gin@nat@height\fi}
\makeatother
% Scale images if necessary, so that they will not overflow the page
% margins by default, and it is still possible to overwrite the defaults
% using explicit options in \includegraphics[width, height, ...]{}
\setkeys{Gin}{width=\maxwidth,height=\maxheight,keepaspectratio}
% Set default figure placement to htbp
\makeatletter
\def\fps@figure{htbp}
\makeatother

\makeatletter
\@ifpackageloaded{caption}{}{\usepackage{caption}}
\AtBeginDocument{%
\ifdefined\contentsname
  \renewcommand*\contentsname{Table of contents}
\else
  \newcommand\contentsname{Table of contents}
\fi
\ifdefined\listfigurename
  \renewcommand*\listfigurename{List of Figures}
\else
  \newcommand\listfigurename{List of Figures}
\fi
\ifdefined\listtablename
  \renewcommand*\listtablename{List of Tables}
\else
  \newcommand\listtablename{List of Tables}
\fi
\ifdefined\figurename
  \renewcommand*\figurename{Figure}
\else
  \newcommand\figurename{Figure}
\fi
\ifdefined\tablename
  \renewcommand*\tablename{Table}
\else
  \newcommand\tablename{Table}
\fi
}
\@ifpackageloaded{float}{}{\usepackage{float}}
\floatstyle{ruled}
\@ifundefined{c@chapter}{\newfloat{codelisting}{h}{lop}}{\newfloat{codelisting}{h}{lop}[chapter]}
\floatname{codelisting}{Listing}
\newcommand*\listoflistings{\listof{codelisting}{List of Listings}}
\makeatother
\makeatletter
\makeatother
\makeatletter
\@ifpackageloaded{caption}{}{\usepackage{caption}}
\@ifpackageloaded{subcaption}{}{\usepackage{subcaption}}
\makeatother
\ifLuaTeX
  \usepackage{selnolig}  % disable illegal ligatures
\fi
\usepackage{bookmark}

\IfFileExists{xurl.sty}{\usepackage{xurl}}{} % add URL line breaks if available
\urlstyle{same} % disable monospaced font for URLs
\hypersetup{
  pdftitle={Informações Pré-Curso},
  colorlinks=true,
  linkcolor={blue},
  filecolor={Maroon},
  citecolor={Blue},
  urlcolor={Blue},
  pdfcreator={LaTeX via pandoc}}

\title{Informações Pré-Curso}
\author{}
\date{}

\begin{document}
\maketitle

Bem vindos à primeira edição do curso ``Análise preditiva de dados
ambientais e geoespacias em R''! O objetivo desse curso é estabelecer
uma base teórica e prática inicial para que vocês possam incorporar o
aprendizado de máquina (\emph{machine learning}) à sua caixa de
ferramentas de pesquisa. Esse curso certamente não irá tornar ninguém um
expert - a ideia pricipal é fornecer o mínimo de fundamentos para que
vocês possam continuar sua jornada de aprendizado de maneira
independente.

Abaixo, algumas informações sobre o curso, e também informações
importantes para preparação \textbf{pré-curso}, incluindo instalação de
software, pacotes, etc.

\subsection{Pré-requisitos:}\label{pruxe9-requisitos}

Familiaridade básica com programação em R. Essa familiaridade pode ser
obtida completando o seguinte vídeo-curso online gratuito:
\url{https://didatica.tech/curso-de-r-online-para-iniciantes/} ou
completando os capítulos `Pré-Requisitos'e `Introdução ao R' deste livro
online gratuito:
\url{https://mauriciovancine.github.io/publications/eco-r/.}

\subsection{Locais,datas, horas:}\label{locaisdatas-horas}

\emph{Dia 09/12}

14:00-17:00h: Auditório do Curso de Biologia

\emph{Dias 10-12/12 e 15-19/12}

A ser determinado

\subsection{Programa (sujeito à
mudança):}\label{programa-sujeito-uxe0-mudanuxe7a}

\begin{itemize}
\item
  Dia 1: Introdução à ciência de dados
\item
  Dia 2: Pré-processamento e visualização de dados tabulares.
\item
  Dia 3 Introdução à modelagem: revisitando os modelos de regressão
  múltipla sob a ótica de predição.
\item
  Dia 4: Aprendizado de dados aplicado à modelagem preditiva com dados
  tabulares.
\item
  Dia 5: Visualização e gerenciamento de dados geoespaciais.
\item
  Dia 6: Aprendizado de dados aplicado à modelagem preditiva com dados
  geoespaciais (classificação de uso e cobertura da terra / modelagem
  quantitativa).
\item
  Dia 7: Prática guiada em modelagem preditiva de dados.
\item
  Dias 8-9: Trabalho supervisionado em projeto independente de modelagem
  preditiva de dados.
\item
  Dia 10: Apresentação de projetos e discussão, encerramento do curso.
\end{itemize}

\textbf{Referências Bibliográficas}

Da Silva FR, Gonçalves-Souza T, Paterno GB, Provete DB, Vancine MH
(2022). Análises ecológicas no R. Nupeea, Recife, PE, Primeira edição.
ISBN 9788579175640.
\url{https://canal6.com.br/livreacesso/livro/analises-ecologicas-no-r/}

Wickham H, Çetinkaya-Rundel M, Grolemund G (2023) R for Data Science.
O'Reilly Media, Inc.~2nd Edition. ISBN: 9781492097402
\url{https://r4ds.hadley.nz/}

Lovelace R, Nowosad J, Muenchow J (2025). Geocomputation with R. The R
Series. CRC Press. ISBN: 1032248882 \url{https://r.geocompx.org/}

Kuhn M, Silge J (2023). Tidy modelling with R: A Framework for Modeling
in the Tidyverse. O'Reilly Media, Inc.~1st Edition. ISBN:1492096482.
\url{https://www.tmwr.org/}

\subsection{Preparação Pré-Curso:}\label{preparauxe7uxe3o-pruxe9-curso}

\begin{itemize}
\item
  Instalar o R versão 4.5.2: \url{https://www.r-project.org/} .
  (\textbf{Atenção:} se você está atualmente usando uma versão anterior
  do R para análises no seu projeto, recomendo instalar a versão 4.5.2
  sem remover sua versão atual - para garantir que nenhuma
  incomplatibilidade de versão seja introduzida e afete o seu trabalho).
\item
  Instalar/atualizar o RStudio para a versão 2025.09.2 Build 418:
  \url{https://posit.co/download/rstudio-desktop/}
\item
  Instalar os pacotes abaixo (copiar e colar o comando em um script):
\end{itemize}

\begin{Shaded}
\begin{Highlighting}[]
\FunctionTok{install.packages}\NormalTok{(}\StringTok{\textquotesingle{}terra\textquotesingle{}}\NormalTok{,}\StringTok{\textquotesingle{}raster\textquotesingle{}}\NormalTok{,}\StringTok{\textquotesingle{}sf\textquotesingle{}}\NormalTok{,}\StringTok{\textquotesingle{}sp\textquotesingle{}}\NormalTok{,}\StringTok{\textquotesingle{}tidyverse\textquotesingle{}}\NormalTok{,}\StringTok{\textquotesingle{}tidymodels\textquotesingle{}}\NormalTok{)}
\end{Highlighting}
\end{Shaded}

\subsection{Material}\label{material}

A versão mais recente do material deste curso ficará disponível em:
\url{https://github.com/thiagosfsilva/ML_course_2025} .



\end{document}
